\chapter*{Závěr}
\addcontentsline{toc}{chapter}{Závěr} % přidá položku závěr do obsahu

V~rámci práce byl vytvořen portál, který umožňuje nahrazení prezenčních Burz škol on-line video konferencemi v~době pandemie koronaviru Covid-19.

Portál byl navržen, vyvinut, nasazen, je používán a vzhledem k~současné pandemické situaci má potenciál zůstat důležitým prvkem i do budoucna.

\emph{TODO: Rozvést}

% V \hyperref[chap:teoretical-part]{teoretické části} jsem popsal prostředky \hyperref[sub:server-architecture]{serverové} a \hyperref[sub:app-architecture]{aplikační architektury} prostředky využívané portálem \bso{} Dále jsem pokryl \hyperref[sub:development-enviroment]{vývojová prostředí}.
% 
% V \hyperref[chap:practical-part]{praktické části} práce jsem popsal jednotlivé kroky, které jsem provedl při vývoji projektu \bso{} Popsal jsem \hyperref[sub:data-model]{návrh datového modelu}, \hyperref[sub:used-technologies]{využíté technologie a důvody k jejich využití}, \hyperref[sub:laravel-development]{vývoj webové aplikace v \acrshort{php} frameworku Laravel}.
% 
% Dále jsem popsal \hyperref[sec:deployment-running]{nasazení a provoz projektu.} V této sekci jsem psal o \hyperref[sub:hosting]{dostupných hostingových službách a zvolené hostingové službě}, \hyperref[sub:deployment]{nasazení aplikace} a \hyperref[sub:update-deployment]{vysazování aktualizací aplikace}.
% 
% V poslední sekci jsem popsal \hyperref[sec:security]{bezpečnostní hrozby} webových aplikací a jejich mitigaci v projektu \bso{}
% 
% Portál samotný je plně funkční. Do portálu se zaregistrovalo 465 středních škol, asi 40\% procent z celkového počtu středních škol v České Republice.

\pagebreak