\section{Serverová infrastruktura}
\label{sub:server-architecture}

Tato sekce popisuje prvky serverové infrastruktury, které jsou nasazeny v~reálném řešení \bso. 

\label{sub:load-balancing}
\subsection{Load Balancery}

Load balancer\cite{load-balancer} je prvek serverové infrastruktury určený k~rozložení příchozích požadavků na skupinu backendových serverů. K~rozložení požadavků je využíváno nejrůznějších algoritmů, jako například \gls{round-robin}, \gls{least-con}, či \gls{ip-hash}. Dělíme je na \acrshort{sw} load balancery a \acrshort{hw} load balancery. 

\acrshort{hw} load balancery jsou zpravidla proprietární a nákladná řešení. Vysokého výkonu dosahují využitím dedikovaných \acrshort{hw} akcelerátorů. Jediné způsoby jejich škálování jsou zakoupení více fyzických strojů, či zakoupení silnějších strojů.

\acrshort{sw} load balancery jsou distribuovány ve formě uživatelských aplikací spustitelných na běžném hardwaru, či ve virtuálních strojích cloudových providerů. Škálovatelnost je tedy velice triviální. V~porovnání s~\acrshort{hw} load balancery stačí pouze \gls{on-demand} zakoupit virtuální stroj s~odpovídajícím výpočetním výkonem.  Tato vlastnost činí z~\acrshort{sw} load balancerů velice flexibilní řešení, ideální pro agilní cloudové prostředí.

\subsection{Objektová úložiště}

Objektové úložiště je typ úložiště určený k~ukládaní nestrukturovaných dat. Na rozdíl od tradičních způsobů ukládání dat, jakými jsou souborové systémy a bloková úložiště, pracují objektová úložiště\cite{object-storage} s~daty jako se samostatnými objekty. Každý objekt obsahuje data samotná, uložená v~jejich nativním formátu, a metadata. Metadata obsahují informace jako přístupová práva objektu, unikátní globální identifikátor a další přidružené informace, jako například formát uložených dat. Díky adresaci za pomocí globálních identifikátorů je možné objekty transparentně přesouvat. Tato vlastnost nám poskytuje nezávislost dat na použitém záznamovém médiu.

Je tedy například možné přesunout málo používané objekty z~diskového úložiště na úložiště založené na magnetické pásce. Na druhou stranu ale také můžeme transparentně přesunout často požadované objekty z~úložiště do pracovní paměti pro rychlý přístup.

Objektová úložiště dovolují ukládání velkého objemu nestrukturovaných dat. Je proto ideální jako úložiště pro uživatelské obrázky, dokumenty a zálohy. Díky své flexibilitě jsou objektová úložiště výrazně levnější než například bloková úložiště.

\subsection{Relační databáze}

Relační databáze\cite{relational-database} jsou úložiště strukturovaných dat, založená na tzv. relačním modelu. Relační model organizuje data do \emph{tabulek}, \expl{\emph{řádků}}{záznamů} a \emph{sloupců}. Vzájemné vztahy mezi záznamy jsou reprezentovány \expl{\emph{relacemi}}{shodnými hodnotami dvou sloupců}. 

Software implementující relační databáze se nazývá \acrfull{rdbms}. \acrshort{rdbms} dále obsahuje implementaci některého z~dotazovacích jazyků, např. \acrshort{sql}, a podporu funkcí jako jsou \acrshort{acid}, \glspl{sql-trigger} a \glspl{sql-saved-procedure}.

\subsubsection{SQL}

Pro dotazování a správu dat uchovaných v~\acrshort{rdbms} se využívá standardizovaný jazyk \acrshort{sql} (\acrlong{sql})\cite{sql}. Díky němu můžeme data analyzovat a získávat z~nich informace. Data můžeme shlukovat, transformovat a zkoumat pomocí nejrůznějších statistických funkcí.

\subsubsection{Spravované databáze}
\label{subsub:managed-databases}

Provize a správa databází je velice komplikovaná. Je třeba řešit replikaci, a také automatické škálování. Spravované databáze\cite{managed-databases} se snaží tyto problémy řešit poskytováním předpřipravených databázových instancí, u~kterých je automaticky řešena provize a škálování. Většina spravovaných databází dále poskytuje velice bezpečné nastavení vyladěné pro maximální bezpečnost a výkon databáze. To nám dovoluje se starat pouze o~databázové objekty a uživatelské účty, tedy aspekty které přímo ovlivňují naši aplikaci.

Další výhodou spravovaných databází je poskytování různých metrik a varování, která nám ukazují detailnější informace o~stavu a provozu databáze. Spravovaná databáze je například schopná detekovat tabulky bez primárního klíče a zaslat upozornění databázovému administrátorovi, či zobrazovat detailní \gls{real-time} statistiky o~propustnosti databáze.

\subsection{Webové servery}

\acrfull{webserver}\cite{webserver} je klíčový prvek infrastruktury webové aplikace. Hlavní rolí \acrshort{webserver}u, jak už název napovídá, je poskytování webových stránek. Moderní \acrshort{webserver}y podporují širokou škálu protokolů jako je \acrfull{http}, \acrfull{ws}, či \acrfull{ssl}. Pro standardní internetovou komunikaci využívají \acrshort{webserver}y síťové porty 80 pro nešifrovanou komunikaci a 443 pro šifrovanou komunikaci. 

Většina \acrshort{webserver}ů poskytuje obsah dvěma způsoby, staticky a dynamicky. Statický obsah je poskytován ze souborů uložených na \acrshort{webserver}u samotném. Dynamický obsah je na druhou stranu generován samostatným aplikačním serverem, který poté využívá \acrshort{webserver} jako komunikační bránu. Některé \acrshort{webserver}y mají schopnost přímo provádět aplikační kód a separátní server není potřeba. Pro vytváření dynamických stránek můžeme například využít \acrshort{php}, \Gls{perl}, či různé \Gls{cgi} skripty. Statický obsah je ideální pro poskytování obsahu jako jsou obrázky, či klientské skripty. Dynamický obsah se na druhou stranu využívá pro obsluhu formulářů, či generování sestav.
