\chapter*{Úvod}
\addcontentsline{toc}{chapter}{Úvod} % přidá položku úvod do obsahu

Každý podzim se v~České Republice konají tzv. výstavy středních škol, nebo také burzy, či veletrhy vzdělávání (názvy se liší kraj od kraje).
  Hromadné akce na kterých se střední školy snaží zaujmout co nejvíce žáků 9.~tříd základních škol a vylíčit svou školu v~nejlepším světle.

Žáci 9.~třídy mají možnost si v~jeden čas a na jednom místě prohlédnout nabídky jednotlivých škol.
Pokud je nějaká škola svými obory zaujme mají možnost získat podrobnější informace od zástupců dané školy. 

Vzhledem k~pandemické situaci roku 2020 spojené s~nemocí COVID\hbox{-}19 nebylo možné tyto Burzy uspořádat. Systém \bso{} umožnil přesunutí celého náboru studentů do virtuálního prostředí internetu.

\section*{Jak to funguje?}
\bso{} je internetový portál zprostředkující kontakt mezi žáky devátých tříd a středními školami za pomoci on-line video konferencí (nejčastěji MS~Teams\cite{ms-teams}, Google~Meet\cite{google-meet}, Zoom\cite{zoom}, \ldots). 
Běžnému návštěvníkovi portál nabízí seznam virtuálních burz škol, ve kterém má každá burza přiřazený termín a lokalitu konání.

Střední školy se mohou registrovat k~jednotlivým burzám a vložit odkaz na svou on-line konferenci. V~den konání burzy se mohou žáci devátých tříd kliknutím na odkaz připojit do on-line konference vybrané školy.

Střední školy mají na portále založený svůj profil s~informacemi o~škole, seznam oborů a další informace jako informační brožura, či výsledky v~soutěžích.

\pagebreak

%Účelem je stručně a věcně seznámit se záměrem a řešením tématu, důvodem jeho volby, stručně nastínit problém, který má být řešen (a proč); cíl práce (co je předmětem řešení, jakým způsobem se postupuje a k~čemu se má dospět -- co bude výsledkem; cíl práce je 
