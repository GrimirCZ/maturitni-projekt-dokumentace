\chapter*{Úvod}
\addcontentsline{toc}{chapter}{Úvod} % přidá položku úvod do obsahu

Každý podzim se v České Republice konají tzv. Výstavy středních škol, nebo také Burzy, či Veletrhy vzdělávání (názvy se liší kraj od kraje). Hromadné akce na kterých se střední školy snaží zaujmout co nejvíce žáků \expl{9. tříd základních škol}{deváťáků} a vylíčit svou školu v nejlepším světle.

Deváťáci mají možnost si v jeden čas a na jednom místě prohlédnout nabídky jednotlivých škol. Pokud je nějaká škola zaujme mají možnost získat podrobnější informace od zástupců dané školy. 

Vzhledem k pandemické situaci roku 2020 spojené s nemocí COVID-19, ale nebylo možné tyto Burzy uspořádat. Systém \bso umožnil přesunutí celého náboru deváťáků do virtuálního prostředí internetu.

\section*{Jak to funguje?}
\bso je internetový portál zprostředkující kontakt mezi deváťáky a středními školami za pomoci on-line video konferencí (nejčastěji MS~Teams\cite{ms-teams}, Google~Meet\cite{google-meet}, Zoom\cite{zoom}, ...). Běžnému návštěvníkovi portál nabízí seznam virtuálních burz škol, ve kterém má každá burza přiřazený termín a lokalitu konání.

Střední školy se mohou registrovat k jednotlivým burzám a vložit odkaz na svou on-line konferenci. V den konání burzy se mohou deváťáci kliknutím na odkaz připojit do on-line konference vybrané školy.

Střední školy mají na portále založený svůj profil s informacemi o škole, seznam oborů a další informace jako informační brožura, či výsledky v soutěžích.

Odkaz na repositář s kódem: \href{https://github.com/GrimirCZ/delta-burza}{https://github.com/GrimirCZ/delta-burza}.

\pagebreak

%Účelem je stručně a věcně seznámit se záměrem a řešením tématu, důvodem jeho volby, stručně nastínit problém, který má být řešen (a proč); cíl práce (co je předmětem řešení, jakým způsobem se postupuje a k~čemu se má dospět -- co bude výsledkem; cíl práce je „páteří“ práce, je jedním z~kritérií i pro posuzování řešení tématu, zpracování práce); dále v~úvodu uvést, jaký je postup řešení, metody; výzkumná otázka a ústřední hypotéza (obvykle u~empirických výzkumů) -- vše formou charakteristiky záměru tématu a postupu jeho řešení (zdůvodnění), rozpracování je pak v~textu práce.

%Úvod
%Každý podzim nastává pro střední školy náborové období budoucích prváků. Školy jezdí na hromadné akce tzv. Výstavy středních škol /Burzy/Veletrhy vzdělávání (názvy se krajově liší), kde se potkávají s deváťáky a líčí svoji školu v nejlepším světle. Deváťáci se mohou v jeden čas a na jednom místě seznámit s kompletní nabídkou škol a u těch, které jsou pro ně zajímavé, zjistit podrobnější informace. Vzhledem k epidemické situaci v roce 2020 spojené s COVIDem bylo jasné, že tento způsob náboru nebude možný. Systém BurzaŠkol.Online umožnil přesunutí celého náboru deváťáků do virtuálního prostředí internetu.
%Jak systém funguje
%BurzaŠkol.Online je internetový portál, který umožňuje spojení mezi deváťáky a středními školami pomocí on-line konferencí (nejčastěji MS Teams, Google Meets, Zoom,…). Základní strukturu portálu tvoří seznam virtuálních burz škol. Každá burza má přiřazený svůj termín (datum a čas konání) a svoji lokalitu působnosti (obvykle okres). Střední školy se registrují k burzám a při registraci vkládají odkaz na svoji on-line konferenci. V termínu konání burzy se deváťáci kliknutím na odkaz připojí do on-line konference vybrané školy.
%Střední školy mají na portále založený svůj profil s marketingovými informacemi o škole a seznam oborů, které nabízejí deváťákům.
%Co jsme dále museli řešit
%Kromě vlastního programování projekt dále musel vyřešit:
%    • Naplnění základních datových číselníků (kraje, okresy, PSČ, obory vzdělávání a jejich kategorizace).
%    • Marketing - jak přesvědčit střední školy a po té základní školy, aby se do projektu zapojily.
%    • Organizaci více jak 70-ti on-line výstav škol ve všech okresech ČR.
%    • Provoz systému (připravit infrastrukturu portálu tak, aby zvládla očekávaný nápor návštěvníků)
%    • Automatizovaný fakturační systém pro vystavovatele
%Další vývoj systému
%Největší slabinou systému se ukázala nízká návštěvnost. Pro její zvýšení jsme portál doplnili o další informace, které jsme získali z veřejně dostupných zdrojů pomocí crawlerů. Konkrétně jsme doplnili:
%    • všechny střední školy (nejen registrované) s jejich obory (zdroj: rejstřík škol MŠMT),
%    • výsledky škol ze soutěží spolupořádaných MŠMT (zdroj: excelence.msmt.cz),
%    • výsledky didaktických testů státních maturit za posledních 5 let (zdroj:vysledky.cermat.cz),
%    • zprávy ČŠI (zdroj: portál ČŠI)
%Plány do budoucna
%Tzv. MVP (minimum valuable product) vznikl za 6 dní a nocí. Systém se i nadále vyvíjel pod obrovským časovým tlakem a je to na něm samozřejmě vidět. Do budoucna plánujeme systém kompletně předělat. Bude dále nabízet možnost on-line burz (může sloužit jako podpůrný nástroj klasických burz a v případě potřeby i jako plnohodnotná náhrada). Zároveň však chceme, aby portál fungoval jako klasický portál škol (např. Atlas školství, stredniskoly.cz,…). Jednáme s největším vydavatelem tištěných publikací pro deváťáky o možnosti spolupráce.
%Do budoucna se chceme stát nejnavštěvovanějším portálem pro vyhledávání středních škol deváťáky s nejrelevantnějšími informacemi.