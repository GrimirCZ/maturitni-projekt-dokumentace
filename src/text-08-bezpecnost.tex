\section{Bezpečnost aplikace}\label{sec:security}

Kapitola zaměřená na zabezpečení projektu \bso{} je z~důvodu komplexity tématu rozdělena na dvě části.
První část pojednává o~zabezpečení systému proti útokům vedeným skrze samotnou aplikaci, např. útokům typu \emph{\acrshort{sql} Injection}.
Druhá kapitola se zaměří na zabezpečení infrastruktury proti útokům vedeným z~internetu cílících na špatně nakonfigurované služby,
např. \emph{Útok na \acrshort{ssh} přihlašování serveru}.

\subsection{Aplikační zabezpečení}

Hlavní nebezpečí při provozování webového portálu, plyne z~práce s~uživatelským vstupem.
Aplikace musí tento vstup přijímat a operovat nad ním. To může vést k~celé řadě útoků.
Je tedy důležité, aby uživatel svým vstupem nemohl webovou aplikaci napadnout, zpomalit, ani jinak vyřadit z~provozu.
Nejdříve se zaměříme na bezpečnost textových polí a útoky s~nimi spojenými.

\subsubsection{SQL Injection}

\acrshort{sql} Injection\cite{sqlinject} je typ útoku, při kterém útočník zadá do textového vstupu kód v~jazyce \acrshort{sql}.
Naše aplikace poté tento kód vezme a přímo ho vloží do řetězce dotazu.
Při vykonávání příkazu pak databázový server vykoná tento škodlivý kód, jako by pocházel od samotné aplikace.

Tento útok představuje velké nebezpečí, jelikož může vést od ztráty dat, až k~úniku citlivých dat do rukou útočníka.

Aplikace \bso{} se proti tomuto typu útoku brání pomocí \expl{\emph{předpřipravených dotazů}\cite{mysqlprepstmt}}{rozdělení dotazu na dvě části}.
První část tvoří příkaz, který popisuje, jak bude daný dotaz proveden.
Druhou část tvoří uživatelské vstupy.
Tato část obsahuje pouze data a databázový server ji nikdy nespouští.
Tím zabraňuje tomu, aby server vykonal uživatelský vstup jako příkaz.
\emph{Předpřipravené dotazy}\cite{mysqlprepstmt} riziko tohoto útoku plně eliminují.

\subsubsection{Cross-site skriptování}

Cross-site skriptování\cite{xss}, nebo také \acrshort{xss}, je typ útoku, při kterém je do textového vstupu zadán validní kód spustitelný prohlížečem. Zranitelný webserver poté tento kód vloží přímo do stránky, která je zaslána na klientské zařízení. Klientské zařízení tento škodlivý kód následně vykoná v~domnění, že se jedná o~kód naší aplikace.

Tento útok představuje pro návštěvníky velké nebezpečí, jelikož může vést ke krádeži uživatelských přihlášení, zobrazování reklam nebo krádeži platebních údajů.
Mezi stránkové skriptování\cite{xss} může být také využito jako mezikrok pro další, např. phishingové\cite{phishing}, útoky.

Potenciálními místy, kde by mohlo u~aplikace \bso{} k~napadení XSS dojít, jsou jednořádková vstupní pole a grafické textové editory. Jednořádková vstupní pole slouží pro jednoduchý uživatelský vstup bez formátování. Zabezpečení je tedy velice jednoduché. Jakýkoliv vstup, který uživatel zadá, je očištěn a veškeré \acrshort{html} značky jsou odstraněny. U~grafických textových editorů však není tato obrana možná, jelikož validním vstupem je i \acrshort{html} kód. Aplikace tedy daný kód na serveru parsuje a odstraní veškeré závadné \acrshort{html} značky, jako je například značka \emph{script}, a závadné \acrshort{html} atributy, jako například \emph{onmove}.

\subsubsection{Podvržení požadavků mezi stránkami}

Podvržení požadavků mezi stránkami (\acrshort{csrf}) \cite{csrf} je typ útoku, při kterém cizí stránka obsahuje odkaz, který vede na naši webovou adresu. Většinou se jedná o~požadavky typu POST\@. Tento požadavek má většinou za cíl vykonat škodlivou akci, např. napsat příspěvek bez vědomí uživatele. Při útoku je využíváno faktu, že prohlížeč s~každým požadavkem odesílá \acrshort{http} cookies, které ověřují uživatele serveru.

Útočník tedy může například vytvořit novou objednávku a tím způsobit finanční újmu. Tento útok představuje pro uživatele velké nebezpečí.

Aplikace \emph{BurzaŠkol.Online} se brání podvržení požadavků tak, že pro každou uživatelskou akci vyžaduje tzv. \acrshort{csrf} Token\cite{csrf}. Token je unikátní pro každý požadavek a dovoluje uživateli vykonat přesně jednu akci. Token je vygenerován serverem, přímo do \acrshort{html} kódu stránky. Útočník k~němu proto nemá přístup a není mu dovoleno vykonat žádnou akci.

\subsubsection{Útok hrubou silou}

Útoky hrubou silou\cite{bruteforce} je skupina útoků využívající velkého počtu požadavků za účelem uhodnutí uživatelských přihlašovacích údajů. Tyto požadavky jsou vedeny na přihlašovací formulář aplikace a podle odpovědi serveru určují zda byl útok úspěšný, či nikoliv.

Útok typu brute-force vede k~vyzrazení přihlašovacích údajů útočníkovi. Útočník tím získá možnost plné kontroly nad uživatelským účtem.

Proti tomuto útoku je nasazena obrana spočívající v~limitaci počtu požadavků na \acrshort{ip} adresu. Pokud uživatel udělá za určený časový více požadavků na přihlášení než je povolený limit, tak aplikace požadavky zahazuje a dále je nezpracovává. Tato ochrana činí útoky typu brute-force velice nepraktické, až nemožné.

\subsection{Zabezpečení infrastruktury}

Dalším důležitým faktorem bezpečnosti, při provozování webového portálu, je zabezpečení serverové infrastruktury.
Útoky na serverovou infrastrukturu aplikace mají často větší dopad než zranitelnosti aplikace samotné.
Tyto útoky mají často za úkol aplikaci vyřadit z~provozu nebo ji zneužít.
Jedná se, např. o~\emph{slovníkové útoky} nebo \emph{bruteforce útoky na \acrshort{ssh} přihlašování}.

\subsubsection{Útoky na SSH přihlašování}

Při útoku na SSH přihlašování serveru se využívá povoleného přihlašování pomocí uživatelských hesel.
Útok se snaží najít kombinace uživatelských jmen a hesel, které nejsou dostatečně komplexní a bezpečné.
Častým cílem útoku je kořenový uživatel, \emph{root}, který by úspěšnému útočníkovi poskytl kompletní kontrolu nad cílovým strojem.
K~prolomení hesel se využívají tzv.\ \expl{slovníkové útoky}{soubory velkého množství častých hesel, kterými se útočník snaží zabezpečení prolomit}.

Pokud dojde k~prolomení \acrshort{ssh} přihlášení serveru získá tím útočník plný přístup k~serverovému terminálu. Je tedy schopný vykonávat všechny akce, ke kterým má napadený účet oprávnění. Útočník může například zapojit stroj do botnetu\cite{botnet}, zfalšovat nebo zaměnit poskytované stránky, či ukrást platební nebo osobní údaje.

Jednou z~hlavních obran, kterou využívají servery projektu \bso{}, proti tomuto útoku je zákaz používaní hesel pro vzdálený přístup.
K~přihlašování se poté využívá tzv. \expl{\acrshort{ssh} klíčů\cite{ssh-keys}}{certifikátů založených na asymetrické kryptografii}.
Druhou linií obrany je poté rozřazení jednotlivých úloh pod samostatné uživatelské účty a princip minimálních oprávnění.
Toto opatření pak minimalizuje škody způsobené útočníkem na úzkou napadenou oblast. 

%\subsubsection{Denial of service útoky}
%
%Útoky typu \acrfull{dos}\cite{denial-of-service} jsou útoky při kterých se útočník snaží zahltit naši infrastrukturu, a tím způsobit její výpadek.
%Nejčastější variantou \acrshort{dos} útoku je \acrfull{ddos} - \acrshort{dos} útok vedený z~mnoha míst najednou \cite{distributed-denial-of-service}. 
%Vyskytují se ve velkém množství variací, např. Slowloris, různé typy floodingu a amplifikační útoky.
%
%\subsubsubsection{Slowloris útoky}
%
%\noindent
%\acrshort{dos} útok Slowloris\cite{slowloris} využívá toho, že každé připojení vyžaduje, tzv. \acrfull{fd}\cite{fd}. \acrshort{fd} je systémový zdroj, který v~unixových systémech reprezentuje otevřené soubory, či síťová připojení. Každý běžící proces má omezený počet \acrshort{fd}, které je schopný otevřít v~jednu chvíli. Cílem útoku Slowloris je donutit webserver k~vyčerpání počtu dostupných \acrshort{fd} tím, že otevře velkou spoustu připojení a zasílá pouze tolik dat, aby server připojení neuzavřel. Snaží se tak napodobit spoustu zařízení s~pomalým připojením. Když dojde k~vyčerpání volných \acrshort{fd} ztratí server schopnost přijímat nové požadavky a na koncového uživatele působí jako přetížený. Hlavní výhoda tohoto útoku je, že nevyžaduje větší prostředky ke svému provedení. Hlavní obranu proti útoku tvoří omezení maximálního množství připojení na \acrshort{ip} adresu. Je také nutné nastavit rozumnou spodní hranici rychlosti připojení.
%
%\subsubsubsection{Floodingové útoky}
%\label{subsec:ddos-flood-attack}
%
%\noindent
%Floodingové \acrshort{ddos} útoky jsou založené na přetížení cílového stroje. Jejich cílem je překročení maximální propustnosti \acrshort{hw} a \acrshort{sw} komponent serveru, a tím způsobit nedostupnost, výpadek, či dokonce fyzické poškození. Tento útok využívá nedokonalosti různých protokolů a jejich implementací. Příkladem floodingových útoků je UDP flood\cite{udp-flood}, SYN flood\cite{syn-flood} nebo HTTP flood\cite{http-flood}.
%
%\subsubsubsection{Amplifikační útoky}
%
%\noindent
%Amplifikační \acrshort{ddos} útoky jsou variantou \hyperref[subsec:ddos-flood-attack]{floodingových útoků} využívající základní internetové služby, jejichž velikost odpovědi je několikanásobně větší než velikost požadavku. U~některých dotazů je poměr až 1:200 \cite{ddos-amplify-ratio}. Hlavními cíli amplifikačních útoků jsou nechráněné \acrshort{ntp} servery\cite{ddos-amplify-ratio} a \acrshort{dns} servery\cite{dns-flood}. Při útoku se využívá faktu, že některé \acrshort{ntp} a \acrshort{dns} servery přijímají požadavky pomocí \acrshort{udp} protokolu, bez kontroly pravosti zpáteční \acrshort{ip} adresy. Útočník zašle malý \acrshort{udp} požadavek s~podvrženou \acrshort{ip} adresou cílového serveru. Zranitelný server poté odpoví na podvrženou \acrshort{ip} adresu a provede \acrshort{udp} flood útok pod záminkou odpovědi na legitimní dotaz.
%
%\subsubsubsection{Obrana proti floodingovým a amplifikačním útoků}
%
%\noindent
%Obrana proti floodingovým a amplifikačním útokům je velice náročná.
%Jeden z~největších problémů je fakt, že škodlivý datový provoz je v~mnohých případech těžce rozeznatelný od normálního datového provozu aplikace.
%Je proto potřeba izolovat útočící \acrshort{ip} adresy od \acrshort{ip} adres legitimních uživatelů.
%Problém s~blokací škodlivého provozu těchto útoků je, že samotný objem blokovaných dat má potenciál naši infrastrukturu přetížit.
%Ideálním, i když nákladným, řešením jsou fyzické firewally, které upravují provoz pomocí hardwarových akcelerátorů. Méně efektivními, ale stále užitečnými, možnostmi je alokace větších serverových prostředků na místa, kde naše služby komunikují s~veřejným internetem, v~kombinaci s~blokováním útočících \acrshort{ip} adres. 
