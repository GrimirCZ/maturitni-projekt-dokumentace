\section{Burzy škol}

Burzy škol jsou tradičním místem, na kterém si žáci 9. ročníků základních škol vybírají své budoucí školy. Burzy jsou konány v podobě veřejných výstav, na které se sjíždějí školy a firmy z širokého okolí. Jednotlivé výstavy se poté konají na podzim, od října do listopadu, aby měli žáci dostatek času na rozhodování před jednotnými přijímacími zkouškami.

Na těchto výstavách mají žáci možnost promluvit si se zástupci jednotlivých škol a firem a zeptat se na všechny důležité informace jako například kritéria přijímacích zkoušek, vyučované předměty a výsledky maturitních zkoušek. Žáci poté mohou tyto vědomosti využít při výběru středního vzdělávání.

Z pohledu vystavovatelů, škol a firem, jde o jednu z největších příležitostí na zviditelnění. Pro začínající a nezavedené školy jsou burzy škol hlavním komunikačním kanálem se žáky devátých ročníků. Firmy se zde snaží se zájemci uzavřít spolupráci ve formě studijních stipendií a následného pracovního místa.

V roce 2021 došlo v České Republice ke zrušení burz škol, z důvodu epidemiologických opatření, přijatých kvůli pandemii koronaviru Covid-19. Zrušení burz znamenalo pro mnoho škol velký problém, jelikož nemohly kontaktovat případné zájemce z řad žáků devátých tříd.