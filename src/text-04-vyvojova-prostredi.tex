\section{Vývojové prostředí}
\label{sub:development-enviroment}

Pro vývoj softwaru je zapotřebí specializovaných vývojových prostředí. Vývojová prostředí obsahují interpreter, či kompilátor (dle typu jazyka)\cite{interpreter-vs-compiler}, textový editor a další pomocný software, např. linter\cite{linter} nebo \mbox{jazykový server\cite{language-server}}. Hlavním účelem vývojových prostředí je umožnit vývoj softwaru a co nejvíce ho zjednodušit. 

Jelikož je příprava vývojových prostředí náročný a zdlouhavý proces, najdeme řadu postupů, jak jej zjednodušit. Existují různé programy, které vývojové prostředí vytvoří a nastaví a snaží se při tom brát v~potaz nejběžnější nastavení, která by běžný uživatel mohl potřebovat. Jedná se o~programy jako \emph{create-react-app}\cite{create-react-app} a \emph{Artisan Console}\cite{laravel-artisan}. Tyto programy nabízejí funkcionality jako zakládání nového projektu, generování šablon pro různé kusy kódu a spouštění projektu.

Hlavní slabinou těchto programů je, že i když pomáhají s~přípravou projektu, často nepomáhají se samotným vývojem. Tento problém se snaží řešit tzv. \acrfull{ide}\cite{ide}.  

\subsection{Integrovaná vývojová prostředí (IDE)}

Hlavním cílem \acrshort{ide} je pokrýt kompletní vývojový cyklus aplikace. Od založení projektu, přes psaní, až po její vydání.
\acrshort{ide} proto obsahuje široké množství nástrojů.
Na rozdíl od pomocných programů se ale jedná o~ucelený balíček.
Dále často obsahují editor kódu s~inteligentním napovídáním a zvýrazňováním, jedna z~hlavních výhod použití \acrshort{ide}.
Samozřejmostí jsou také pokročilé možnosti ladění kódu s~\acrshort{gui} \glspl{debugger} nebo navigování pomocí symbolů.

Pokud by nám základní funkcionalita \acrshort{ide} nestačila, můžeme využít dalších zásuvných modulů.
Ty jsou poskytovány buďto autorem samotného \acrshort{ide} nebo třetí stranou.
Zásuvné moduly poskytují funkcionality jako podporu dalších jazyků, či formátů souborů, rozšíření možností editování textu nebo integraci s~dalšími službami.
