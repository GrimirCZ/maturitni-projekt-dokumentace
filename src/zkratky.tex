\newacronym{ssh}{SSH}{Secure Shell}
\newacronym{dos}{DoS}{Denial of Service}
\newacronym{ddos}{DDoS}{Distributed Denial of Service}
\newacronym{fd}{FD}{File Descriptor}
\newacronym{ip}{IP}{Internet Protocol}

\newacronym{hw}{HW}{Hardware}
\newacronym{sw}{SW}{Software}

\newacronym{dns}{DNS}{Domain Name System}
\newacronym{ntp}{NTP}{Network Time Protocol}

\newacronym{udp}{UDP}{User Datagram Protocol}

\newacronym{sql}{SQL}{Structured Query Language}

\newacronym{webserver}{webserver}{Webový server}
\newacronym{http}{HTTP}{Hyper Text Transport Protocol}
\newacronym{html}{HTML}{Hyper Text Markup Language}
\newacronym{css}{CSS}{Cascading Stylesheets}
\newacronym{js}{JS}{JavaScript}
\newacronym{ws}{WS}{Web Sockets}
\newacronym{ssl}{SSL}{Secure Sockets Layer}
\newacronym{wysiwyg}{WYSIWYG}{What You See Is What You Get}

\newacronym{stdout}{STDOUT}{Standard Output}
\newacronym{stdin}{STDIN}{Standard Input}
\newacronym{stderr}{STDERR}{Standard Error}

\newacronym{php}{PHP}{Rekurzivní název programovacího jazyka PHP: Hypertext Preprocessor}

\newacronym{ide}{IDE}{Integrovaná vývojová prostředí}

\newacronym{gui}{GUI}{Grafické uživatelské rozhraní}

\newacronym{http-sse}{SSE}{Server-sent events}

\newacronym{msmt}{MŠMT}{Ministerstvo školství, mládeže a tělovýchovy České republiky}

\newacronym{aws}{AWS}{Amazon Web Services}
\newacronym{aws-ses}{SES}{Simple Email Service}

\newacronym{dmz}{DMZ}{Demilitarized zone}

\newacronym{smtp}{SMTP}{Simple Mail Transfer Protocol}
\newacronym{imap}{IMAP}{Internet Message Access Protocol}
\newacronym{wss}{WSS}{WebSocket Secure}

\newacronym{csrf}{CSRF}{Cross-site request forgery}
\newacronym{xss}{XSS}{Cross-site scripting}

\newglossaryentry{ci-cd}{name={CI/CD}, description={Continuous integration and continuous delivery. Mechanismus pro automatické testování a nasazování softwaru}}

\newglossaryentry{dkim}{name={DKIM}, description={DomainKeys Identified Mail. Systém certifikátů pro ověření autenticity odesílatele emailu}}

\newglossaryentry{scaffolding}{name={scaffolding}, description={Předpřipravená souborová struktura pro vývoj projektu}}

\newglossaryentry{perl}{name={Perl}, description={Rodina dvou vyšších dynamických programovacích jazyků Perl 5 a Perl 6}}

\newglossaryentry{cgi}{name={CGI}, description={Common Gateway Interface. Rozhraní umožňující webovým serverům spouštět terminálové aplikace}}

\newglossaryentry{debugger}{name={Debugger}, plural={debuggery}, description={Programy určené pro ladění programů}}

\newglossaryentry{round-robin}{name={Round Robin}, description={Algoritmus sekvenčně rotující backendové servery ze skupiny dostupných serveru, tj.\ jeden po druhém}}
\newglossaryentry{least-con}{name={Least Connections}, description={Algoritmus vybírající server podle nejnižšího počtu právě zpracovávaných požadavků}}
\newglossaryentry{ip-hash}{name={IP Hash}, description={Algoritmus využívající klientskou IP adresu pro vybrání backendového serveru}}
\newglossaryentry{on-demand}{name={on-demand}, description={Ve chvili potřeby. Typ škálování, při kterém se daný zdroj automaticky přizpůsobuje objemu uživatelských požadavků}}

\newglossaryentry{real-time}{name={real-time}, description={V reálném čase. Zobrazování, či provádění akcí, v~současnou chvíli nebo s~minimální odezvou}}

\newglossaryentry{open-source}{name={open-source}, description={Software s~otevřeným zdrojovým kódem}}

\newglossaryentry{mvp}{name={MVP}, description={Minimum Viable Product. Nejmenší možná verze produktu, která je schopná plnit zadaný účel}}

\newglossaryentry{pubsub}{name={pub-sub}, description={Publish-subscribe. Způsob předávání zpráv. Zprávy jsou organizovány do kanálů. Klienti se do těchto kanálů mohou zapojit a dostávat příslušné zprávy}}

\newglossaryentry{framework}{name={framework}, description={Předpřipravený základ pro tvorbu aplikací}, plural={frameworky}}

\newglossaryentry{orm}{name={ORM}, description={Object-Relational Mapper. Software sloužící pro konverzi dat mezi relačním a objektovým modelem}}
\newglossaryentry{php-router}{name={směrovač}, description={Software rozhodující o~tom, jaký kód bude spuštěn v~odpovědi na uživatelský požadavek}}
\newglossaryentry{php-templater}{name={šablonovací knihovna}, description={Knihovna ulehčující generování HTML}}
\newglossaryentry{bug}{name={Bug}, description={Chyby programu zapříčiněné chybou, či nepozorností programátora}, plural={bugy}}
\newglossaryentry{http-polling}{name={HTTP Polling}, description={Periodické stahování dat za pomoci separátních HTTP požadavků.}, plural={bugy}}


%%% TODO: What to use?
\newacronym{rdbms}{RDBMS}{Relational Database Management System}
%%%\newacronym{rdbms}{SŘRBD}{Systém řízení relační báze dat}

\newglossaryentry{dql}{name={DQL},description={Jazyk sloužící pro vytváření dotazů nad daty. DQL je někdy zmiňováno jako součást \gls{dml}},first={Data Query Language (DQL)}}
\newglossaryentry{dml}{name={DML},description={Jazyk sloužící pro manipulaci, vytváření a mazání dat},first={Data Manipulation Language (DML)}}
\newglossaryentry{dcl}{name={DCL},description={Jazyk sloužící pro řízení přístupu k~datům},first={Data Control Language (DCL)}}
\newglossaryentry{ddl}{name={DDL},description={Jazyk sloužící pro definování struktury dat a ostatních databázových objektů},first={Data Definition Language (DDL)}}
\newglossaryentry{tcl}{name={TCL},description={Jazyk sloužící pro vytváření a správu transakcí},first={Transaction Control Language (TCL)}}

\newacronym{acid}{ACID}{Atomicity, consistency, isolation, durability}

\newglossaryentry{sql-trigger}{name={Trigger},plural={triggery},description={\acrshort{sql} kód spuštěný v~návaznosti ná různé události}}
\newglossaryentry{sql-saved-procedure}{name={Uložená procedura},plural={uložené procedury},description={Pojmenovaný \acrshort{sql} kód uložený v~\acrshort{rdbms}, který můžeme opakovaně spouštět}}

\newglossaryentry{crawler}{name={Crawler},plural={crawlery},description={Program určený k stahování dat z webových stránek}}
